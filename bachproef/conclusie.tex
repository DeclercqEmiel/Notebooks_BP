%%=============================================================================
%% Conclusie
%%=============================================================================

\chapter{Conclusie}
\label{ch:conclusie}

% TODO: Trek een duidelijke conclusie, in de vorm van een antwoord op de
% onderzoeksvra(a)g(en). Wat was jouw bijdrage aan het onderzoeksdomein en
% hoe biedt dit meerwaarde aan het vakgebied/doelgroep? 
% Reflecteer kritisch over het resultaat. In Engelse teksten wordt deze sectie
% ``Discussion'' genoemd. Had je deze uitkomst verwacht? Zijn er zaken die nog
% niet duidelijk zijn?
% Heeft het onderzoek geleid tot nieuwe vragen die uitnodigen tot verder 
%onderzoek?

Wanneer er wordt gewerkt met cross-validation en de Mean Average Error wordt gebruikt als foutmaat. Zullen de volgende modellen de beste voorspellingen kunnen maken van de evolutie van het ijsoppervlak op de polen:
\begin{itemize}
    \item Bij de univariate niet-seizoensgebonden tijdreeks zal het ARIMA-model het beste resultaat behalen.
    \item Bij de univariate seizoensgebonden tijdreeks presteert het SARIMA-model met random walk differentiatie het best.
    \item Bij de multivariate niet-seizoensgebonden tijdreeks behaalt het VARMAX-model de beste resultaten.
    \item Bij de multivariate seizoensgebonden tijdreeks zal het LSTM model de beste prestatie leveren.\\
\end{itemize}


Mogelijke onderwerpen voor verder onderzoek zouden dieper in kunnen gaan op gevallen met externe regressoren waarbij bepaalde features in de tijdreeks ook in de testset meegegeven worden. Hiervoor zijn namelijk ingebouwde opties bij ARIMA- en Prophet-modellen. Daarnaast zou een uitgebreidere studie van LSTM-modellen bij deze data interessant kunnen zijn aangezien ze in deze bachelorproef slechts beperkt aan bod komen.
