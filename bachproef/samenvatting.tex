%%=============================================================================
%% Samenvatting
%%=============================================================================

% TODO: De "abstract" of samenvatting is een kernachtige (~ 1 blz. voor een
% thesis) synthese van het document.
%
% Deze aspecten moeten zeker aan bod komen:
% - Context: waarom is dit werk belangrijk?
% - Nood: waarom moest dit onderzocht worden?
% - Taak: wat heb je precies gedaan?
% - Object: wat staat in dit document geschreven?
% - Resultaat: wat was het resultaat?
% - Conclusie: wat is/zijn de belangrijkste conclusie(s)?
% - Perspectief: blijven er nog vragen open die in de toekomst nog kunnen
%    onderzocht worden? Wat is een mogelijk vervolg voor jouw onderzoek?
%
% LET OP! Een samenvatting is GEEN voorwoord!

%%---------- Nederlandse samenvatting -----------------------------------------
%
% TODO: Als je je bachelorproef in het Engels schrijft, moet je eerst een
% Nederlandse samenvatting invoegen. Haal daarvoor onderstaande code uit
% commentaar.
% Wie zijn bachelorproef in het Nederlands schrijft, kan dit negeren, de inhoud
% wordt niet in het document ingevoegd.

\IfLanguageName{english}{
\selectlanguage{dutch}
\chapter*{Samenvatting}
}{}

%%---------- Samenvatting -----------------------------------------------------
% De samenvatting in de hoofdtaal van het document

\chapter*{\IfLanguageName{dutch}{Samenvatting}{Abstract}}

In deze bachelorproef zullen verschillende voorspellingstechnieken voor verschillende types tijdreeksen geanalyseerd worden. Het onderscheid tussen deze types wordt gemaakt op basis van 2 criteria namelijk seizoensgebondenheid en het aantal onafhankelijke variabelen. In totaal zullen er 4 types tijdreeksen getest worden hoe deze precies opgesteld worden zal hieronder opgesomd worden.
\begin{itemize}
    \item Tijdreeksen met enkel de tijd als onafhankelijke variabele en 1 afhankelijke variabele zonder seizoensgebonden verband
    \item Tijdreeksen met enkel de tijd als onafhankelijke variabele en 1 afhankelijke variabele met een seizoensgebonden verband
    \item Tijdreeksen met 2 onafhankelijke variabelen waarvan 1 de tijd en 1 afhankelijke variabele zonder een seizoensgebonden verband
    \item Tijdreeksen met 2 onafhankelijke variabelen waarvan 1 de tijd en 1 afhankelijke variabele met een seizoensgebonden verband
\end{itemize} De eerste techniek die zal toegepast worden is een
 ARIMA/VARMAX model. Als tweede zal een  recurrent neuraal netwerk van het type LSTM (Long Term Short Memory) op dezelfde data toegepast worden en tenslotte zal ook een Prophet-model gebruikt worden om voorspellingen te maken. Voor elk van deze 4 types tijdreeksen werd bepaald welk modeltype de laagste MAE behaalde bij gebruik van cross validation. Dit werd toegepast een dataset van de poolijsdikte waaruit deze resultaten behaald werden:
 \begin{itemize}
     \item Bij de univariate niet-seizoensgebonden tijdreeks zal het ARIMA-model het beste resultaat behalen.
     \item Bij de univariate seizoensgebonden tijdreeks presteert het SARIMA-model met random walk differentiatie het best.
     \item Bij de multivariate niet-seizoensgebonden tijdreeks behaalt het VARMAX-model de beste resultaten.
     \item Bij de multivariate seizoensgebonden tijdreeks zal het LSTM model de beste prestatie leveren.
 \end{itemize}

Tenslotte dient zeker nog vermeld te worden dat dit resultaat zal afhangen van tijdreeks tot tijdreeks en deze modellen niet bij elke tijdreeks het meest optimale resultaat zullen behalen.
