%%=============================================================================
%% Inleiding
%%=============================================================================

\chapter{\IfLanguageName{dutch}{Inleiding}{Introduction}}
\label{ch:inleiding}

%De inleiding moet de lezer net genoeg informatie verschaffen om het onderwerp te begrijpen en in te zien waarom de onderzoeksvraag de moeite waard is om te onderzoeken. In de inleiding ga je literatuurverwijzingen beperken, zodat de tekst vlot leesbaar blijft. Je kan de inleiding verder onderverdelen in secties als dit de tekst verduidelijkt. Zaken die aan bod kunnen komen in de inleiding~\autocite{Pollefliet2011}:

De hoeveelheid data die men kan verwerven in het digitale tijdperk waarin we de dag van vandaag leven is gigantisch zeker met de opkomst van het internet der dingen beter gekend onder de noemer \textit{Internet of Things}. Deze data kan enorm uiteenlopend zijn, zo wordt onder andere de buitentemperatuur bijgehouden, maar ook de waarde van de aandelen van een bedrijf op de beurs, het aantal bezette plaatsen in een parking, het aantal mensen dat positief getest heeft op het coronavirus, ...

Zo kan het lijstje nog een hele tijd aangevuld worden, maar de net opgenoemde gegevens zijn niet enkel voorbeelden van data. Deze zaken vari\"{e}ren ook nog eens doorheen de tijd. Om dit in contrast te stellen met een ander voorbeeld zou een naam of een postcode van je geboorteplaats niet vari\"{e}ren en dus niet tijdsgebonden zijn. Een sequentie van data die tijdsgebonden is wordt benoemd als een tijdreeks ofwel een \textit{time series}. 

Een voorbeeld hiervan zou het aantal dagelijks gewandelde kilometers van het afgelopen jaar zijn. Doordat dit een tijdreeks is zouden we het aantal dagelijks gewandelde kilometers van het volgende jaar kunnen voorspellen met behulp van bepaalde modellen. Daaruit zouden we dan bijvoorbeeld kunnen vaststellen dat er in de winter minder gewandeld zal worden. Het nut hiervan zou dan kunnen zijn dat men inziet dat men te weinig lichaamsbeweging zal hebben en daar dan zal op inspelen door een indoorsport te beoefenen zodat men toch nog voldoende lichaamsbeweging heeft. Bij dit voorbeeld is het verband zeer simpel en kan men zich afvragen waarvoor het opstellen van een model nuttig zou zijn aangezien dit vrij makkelijk af te leiden valt met het blote oog. Maar soms zal het verband een pak complexer zijn dan een seizoen waardoor het moeilijk te vatten valt voor het menselijk brein maar praktischer is om te identificeren door middel van een wiskundig model. 

De modellen die onderzocht zullen worden zijn: ARIMA/SARIMAX/VARMAX-modellen, LSTM-modellen en Prophet-modellen. Daarnaast zal ook nog rekening gehouden met het aantal invoerparameters. Zo zullen modellen die gebruik maken van 1 invoerparameter benoemd worden als univariate modellen, modellen die gebruik maken van meerdere invoerparameters daarentegen zullen benoemd worden als multivariate modellen. Deze 2 types modellen kunnen dan nog eens onderverdeeld worden in seizoensgebonden data en niet-seizoensgebonden modellen. 

\section{\IfLanguageName{dutch}{Probleemstelling}{Problem Statement}}
\label{sec:probleemstelling}

Voorspelde tijdreeksen kunnen zeer belangrijke data zijn om de richting waarin bepaalde beslissingen genomen moeten worden aan te geven. Een brandend actueel voorbeeld hiervan zou een voorspellingsmodel van de coronacijfers zijn wanneer er zo'n model beschikbaar is met een betrouwbaarheid van 100\% wat zou er in principe geen discutie meer mogen zijn over de ernst van de situatie en het treffen van de correcte maatregelen zou een pak praktischer zijn. Dit is echter vrij onwaarschijnlijk aangezien er in dit geval tal van factoren een invloed hebben op die cijfers waarvan er velen zeer moeilijk te registeren vallen. 
Gelukkig zijn er ook problemen waarvan de factoren makkelijker registreerbaar zijn zoals het verminderen van de ijsoppervlakten aan de polen. Zo zal onderzocht worden welk model de beste voorspellingen maakt voor univariate niet-seizoensgebonden tijdreeksen, univariate seizoensgebonden tijdreeksen, multivariate niet-seizoensgebonden tijdreeksen en multivariate seizoensgebonden tijdreeksen. De conclusies en de broncode van dit onderzoek zal kan kunnen dienen als basis voor het opstellen van voorspellingsmodellen van andere tijdreeksen.


\section{\IfLanguageName{dutch}{Onderzoeksvraag}{Research question}}
\label{sec:onderzoeksvraag}

Vergelijkende studie van ARIMA-, LSTM- en Prophet-voorspellingsmodellen van univariate en multivariate, seizoensgebonden en niet-seizoensgebonden tijdreeksen.

\section{\IfLanguageName{dutch}{Onderzoeksdoelstelling}{Research objective}}
\label{sec:onderzoeksdoelstelling}

De doelstelling van deze bachelorproef is om te achterhalen welk model van de volgende drie:
\begin{itemize}
    \item Autoregressie
    \item LSTM
    \item Prophet
\end{itemize}
 de beste resultaten halen voor data van een voorbeeldtijdreeks waarbij een of meerdere onafhankelijke variabelen beschikbaar zijn en/of een seizoensgebonden verband aanwezig is. Dit onderzoek zal ook kunnen dienen als basis voor het opstellen van een voorspellingsmodel van andere tijdreeksen. \\
 
 Er dient echter wel vermeld te worden dat deze resultaten altijd zullen vari\"{e}ren van tijdreeks tot tijdreeks en een optimaal model voor de ene tijdreeks niet altijd de beste keuze zal zijn voor een andere tijdreeks.

\section{\IfLanguageName{dutch}{Opzet van deze bachelorproef}{Structure of this bachelor thesis}}
\label{sec:opzet-bachelorproef}

% Het is gebruikelijk aan het einde van de inleiding een overzicht te
% geven van de opbouw van de rest van de tekst. Deze sectie bevat al een aanzet
% die je kan aanvullen/aanpassen in functie van je eigen tekst.

De rest van deze bachelorproef is als volgt opgebouwd:

In Hoofdstuk~\ref{ch:stand-van-zaken} wordt een overzicht gegeven van de stand van zaken binnen het onderzoeksdomein, op basis van een literatuurstudie.

In Hoofdstuk~\ref{ch:methodologie} wordt de methodologie toegelicht en worden de gebruikte onderzoekstechnieken besproken om een antwoord te kunnen formuleren op de onderzoeksvragen.

% TODO: Vul hier aan voor je eigen hoofstukken, één of twee zinnen per hoofdstuk

In Hoofdstuk~\ref{ch:conclusie}, tenslotte, wordt de conclusie gegeven en een antwoord geformuleerd op de onderzoeksvragen. Daarbij wordt ook een aanzet gegeven voor toekomstig onderzoek binnen dit domein.